\section{Test Document for Semantic Blocks}

This document tests all 7 semantic marker types.

\subsection{Diagram Needs}

\begin{needsdiagram}
Timing diagram showing clock relationships between cog and hub memory access cycles.
\end{needsdiagram}

\subsection{Preliminary Content}

\begin{preliminarycontent}
This feature documentation is under development. The API may change.
\end{preliminarycontent}

\subsection{Verification Needed}

\begin{needsverification}
The frequency calculation formula needs hardware verification on actual P2 chips.
\end{needsverification}

\subsection{Examples Required}

\begin{needsexamples}
Code examples needed for ADC mode configuration with different sampling rates.
\end{needsexamples}

\subsection{Technical Review}

\begin{needstechreview}
Smart Pin transition mode behavior requires review by hardware engineering team.
\end{needstechreview}

\subsection{Code Review}

\begin{needscodereview}
This PASM2 routine needs optimization review for cycle count and register usage.

\begin{verbatim}
        rdpin   temp, #PIN_ADC
        shr     temp, #16
        mov     result, temp
\end{verbatim}
\end{needscodereview}

\subsection{Tips}

\begin{tipbox}
Use Smart Pins for hardware timing whenever possible - they're more accurate than software delays.
\end{tipbox}

\subsection{Regular Content}

This is normal paragraph text that should remain unchanged.